% ============================================================================================
% BAB III ANALISIS MASALAH
% Pembagian subbab tidak rigid dan dapat bervariasi. Bab ini minimal berisi analisis kebutuhan
% fungsional dan nonfungsional, analisis berbagai alternatif solusi yang dapat ditawarkan, dan
% metode pemilihan solusi yang diusulkan.
% ============================================================================================
\chapter{ANALISIS MASALAH}
\label{chap:analisis-masalah}
\section{Analisis Kondisi Saat Ini}
Sistem deteksi \textit{deepfake} yang ada saat ini dapat dimodelkan sebagai sebuah alur proses yang bergantung penuh pada ketersediaan data pelatihan. Gambar \ref{gambar:III-1} menggambarkan model konseptual dari alur kerja deteksi \textit{deepfake} yang umum digunakan saat ini di Indonesia.
\begin{figure}[H]
    \centering
    \includegraphics[width=0.7\textwidth]{image/GambarIII.png}
    \caption{Model Konseptual Alur Deteksi \textit{Deepfake} Saat Ini}
    \label{gambar:III-1}
\end{figure}

Alur kerja dari model konseptual dimulai dari \textit{dataset} publik yang telah banyak ditinjau dalam studi literatur, seperti FaceForensics , Celeb-DFv2 , dan DFDC \autocite{sohan2023}. Sebagian besar dataset ini memiliki karakteristik utama berupa wajah non-Indonesia dan video dengan resolusi yang masih tergolong tinggi. Dataset ini yang kemudian digunakan untuk proses pelatihan , dilanjutkan dengan arsitektur \textit{deep learning} belajar mengenali pola dan artefak manipulasi. Hasil dari proses ini adalah sebuah model deteksi yang menunjukkan performa sangat tinggi di lingkungan laboratorium.

Namun, seperti yang telah diidentifikasi dalam studi literatur , masalah fundamental muncul apabila model tersebut digunakan untuk mendeteksi \textit{deepfake} dalam konteks dunia nyata di Indonesia. Video-video di Indonesia berisi orang-orang lokal (Indonesia) dan, yang lebih penting, mengalami kompresi agresif akibat penyebaran di sosial media serta \textit{bandwidth} internet yang tidak merata.

Kondisi ini sejalan dengan temuan \autocite{montibeller2025}, yang menyatakan bahwa kompresi video secara signifikan merusak fitur forensik tingkat rendah, sehingga menyulitkan detektor. Akibat dari kesenjangan data (distribusi \textit{out-of-distribution}) antara data latih (resolusi tinggi, non-Indonesia) dengan data di lapangan (resolusi rendah, Indonesia) , performa model yang sebelumnya tinggi dapat menurun drastis. Kondisi inilah yang menjadi justifikasi utama perlunya pengembangan \textit{dataset} baru yang dapat menjembatani kesenjangan tersebut.

\section{Analisis Kebutuhan}
Berikut ini adalah penjabaran terkait kebutuhan-kebutuhan untuk membuat \textit{dataset} yang baru.

\subsection{Identifikasi Masalah Pengguna}
Pengguna sistem ini adalah Steven (pembuat model) dan peneliti lain di bidang deteksi \textit{deepfake}. Masalah yang dihadapi oleh para pengguna adalah:
\begin{enumerate}
    \item Steven (pembuat model)
        \begin{enumerate}[label=\alph*.]
            \item Steven membutuhkan data pelatihan yang memadai untuk membuat model yang \textit{robust} terhadap video berkualitas rendah di Indonesia.
            \item Modelnya mungkin akan \textit{overfitting} pada video berkualitas tinggi dan gagal di skenario dunia nyata Indonesia.
        \end{enumerate}
    \item Peneliti lainnya
        \begin{enumerate}[label=\alph*.]
            \item Peneliti tidak memiliki \textit{benchmark dataset} untuk menguji validitas model deteksi \textit{deepfake} mereka dalam konteks Indonesia.
        \end{enumerate}
\end{enumerate}

Untuk mencari solusi atas masalah-masalah tersebut, perlu disusun kebutuhan fungsional dan nonfungsional sistem yang diperlukan. Subbab berikut menjabarkan kebutuhan-kebutuhan tersebut.

\subsection{Kebutuhan Fungsional}
Berikut ini adalah Tabel \ref{tab:kebutuhan-fungsional} yang berisi kebutuhan-kebutuhan fungsional yang dibutuhkan untuk menjawab masalah-masalah tersebut.

\begin{table}[H] 
    \centering
    \caption{Kebutuhan Fungsional \textit{Dataset}}
    \label{tab:kebutuhan-fungsional}
    \renewcommand{\arraystretch}{1.2} % Mengatur jarak antar baris
    \resizebox{\textwidth}{!}{%
        \begin{tabular}{|l|l|p{10cm}|} % Lebar kolom deskripsi diperbesar agar proporsional
            \hline
            \textbf{Kode} & \textbf{Kebutuhan Fungsional} & \textbf{Deskripsi} \\
            \hline
            F-01 & Konteks Lokal & \textit{Dataset} harus berisi video asli yang menampilkan wajah orang Indonesia yang beragam. \\
            \hline
            F-02 & Representasi Etnis & \textit{Dataset} harus merepresentasikan keragaman suku bangsa utama di Indonesia. \\
            \hline
            F-03 & Teknik Relevan & \textit{Dataset} harus berisi video palsu yang dibuat menggunakan teknik \textit{deepfake} yang umum, seperti visual \textit{deepfake}, audio \textit{deepfake}, maupun \textit{multimodal deepfake}. \\
            \hline
            F-04 & Simulasi Kompresi & \textit{Dataset} harus menyertakan versi video (asli dan palsu) yang telah disimulasikan efek kompresinya untuk meniru kualitas rendah di media sosial. \\
            \hline
            F-05 & Pelabelan Jelas & Setiap video dalam \textit{dataset} harus memiliki label yang jelas. \\
            \hline
            F-06 & Metadata Lengkap & \textit{Dataset} harus disertai \textit{metadata} yang detail untuk setiap video palsu. \\
            \hline
            F-07 & Keseimbangan Data & \textit{Dataset} harus memiliki keseimbangan yang wajar antara jumlah video asli dan palsu, serta antara video berkualitas tinggi dan rendah. \\
            \hline
        \end{tabular}%
    }
\end{table}

\subsection{Kebutuhan Nonfungsional}
Berikut ini adalah Tabel \ref{tab:kebutuhan-non-fungsional} yang berisi atribut-atribut mengenai bagaimana (\textit{HOW}) sistem bekerja.

\begin{table}[H] 
    \centering
    \caption{Kebutuhan Non Fungsional \textit{Dataset}}
    \label{tab:kebutuhan-non-fungsional}
    \renewcommand{\arraystretch}{1.2} % Mengatur jarak antar baris
    \resizebox{\textwidth}{!}{%
        \begin{tabular}{|l|l|p{10cm}|} % Lebar kolom penjelasan diperbesar agar proporsional
            \hline
            \textbf{Kode} & \textbf{Kebutuhan Non Fungsional} & \textbf{Penjelasan} \\
            \hline
            NF-01 & Relevansi & \textit{Dataset} harus sangat relevan dengan masalah disinformasi \textit{deepfake} di Indonesia. \\
            \hline
            NF-02 & Keberagaman & \textit{Dataset} harus beragam, tidak hanya dari segi etnis, tetapi juga pencahayaan, latar belakang, dan usia subjek. \\
            \hline
            NF-03 & Validitas & Label harus 100\% akurat. Proses validasi harus memastikan tidak ada kesalahan pelabelan. \\
            \hline
            NF-04 & Realisme & Video berkualitas rendah yang dihasilkan harus secara akurat mencerminkan artefak kompresi yang ditemukan di platform nyata. \\
            \hline
        \end{tabular}%
    }
\end{table}

\section{Analisis Pemilihan Solusi}
Berikut ini adalah beberapa alternatif solusi yang diusulkan beserta metode dalam penentuan solusi.

\section{Analisis Pemilihan Solusi}
Berikut ini adalah beberapa alternatif solusi yang diusulkan beserta metode dalam penentuan solusi.

\subsection{Penjabaran Alternatif Solusi}
Terdapat tiga alternatif solusi utama yang diidentifikasi untuk memenuhi kebutuhan \textit{dataset} \textit{deepfake} berkualitas rendah dalam konteks Indonesia.

\subsubsection{Alternatif Solusi 1: Kompresi \textit{Dataset} Publik yang Ada}
Solusi ini mengusulkan penggunaan \textit{dataset} publik yang sudah mapan dan teruji, seperti FaceForensics++ atau Celeb-DF, sebagai basis. Pekerjaan utama dalam alternatif ini adalah mengambil video-video (asli dan palsu) dari \textit{dataset} tersebut dan kemudian menerapkan simulasi kompresi video secara sistematis, sebagaimana yang diuraikan oleh \autocite{montibeller2025}. Pendekatan ini secara langsung menargetkan kebutuhan F-04 (Simulasi Kompresi) dan NF-04 (Realisme Kompresi).

Kelebihan utama dari pendekatan ini adalah efisiensi waktu dan sumber daya. Proses pengumpulan data (Fase Pengumpulan) dan generasi \textit{deepfake} (Fase Generasi) dapat dilewati seluruhnya. Mengingat proses generasi \textit{deepfake} sangat intensif secara komputasi dan memakan waktu (Risiko 3), alternatif ini sangat menghemat sumber daya. \textit{Dataset} yang ada juga sudah memiliki pelabelan yang jelas (F-05) dan dibuat dengan teknik yang relevan (F-03).

Namun, solusi ini memiliki kelemahan fundamental yang tidak dapat diabaikan. \textit{Dataset} yang dihasilkan akan gagal total dalam memenuhi kebutuhan inti dari penelitian ini, yaitu F-01 (Konteks Lokal) dan F-02 (Representasi Etnis). Model yang dilatih pada \textit{dataset} ini (misalnya, wajah Kaukasia berkualitas rendah) kemungkinan besar akan tetap gagal melakukan generalisasi ketika dihadapkan pada video \textit{deepfake} dengan wajah Indonesia berkualitas rendah. Kegagalan ini terjadi karena model tidak hanya belajar dari artefak kompresi, tetapi juga dari fitur-fitur wajah yang spesifik secara demografis. Selain itu, \textit{metadata} yang ada (F-06) mungkin tidak cukup lengkap untuk tujuan penelitian, karena kita tidak memiliki kontrol atas parameter generasi \textit{deepfake} aslinya.

\subsubsection{Alternatif Solusi 2: Pengumpulan Pasif \textit{Deepfake} Beredar}
Solusi ini berfokus pada pengumpulan video \textit{deepfake} yang sudah beredar secara "liar" (\textit{in-the-wild}) di platform media sosial Indonesia. Pendekatan ini memiliki keunggulan teoretis tertinggi dalam hal relevansi (NF-01). \textit{Dataset} yang dihasilkan akan terdiri dari video-video yang memang menjadi masalah nyata di lapangan, sehingga secara otomatis memenuhi F-01 (Konteks Lokal) dan F-04 (Simulasi Kompresi), karena video tersebut adalah produk kompresi dunia nyata.

Meskipun demikian, solusi ini memiliki banyak ketidakpastian fatal yang membuatnya tidak cocok untuk penelitian sistematis.
\begin{enumerate}
    \item {Ketiadaan \textit{Ground Truth}}: Hampir tidak mungkin mendapatkan video asli (\textit{ground truth}) untuk setiap video palsu yang ditemukan. Tanpa video asli, pelabelan (F-05) menjadi tidak terkontrol dan tidak dapat divalidasi 100\%.
    \item {Metadata Tidak Diketahui}: Teknik \textit{deepfake} yang digunakan (F-03) dan \textit{metadata} video (F-06) tidak akan diketahui. Kita tidak tahu alat apa yang digunakan, parameter apa, atau berapa kali video itu dikompresi ulang. Ini menjadikannya \textit{dataset} "kotak hitam" (\textit{black box}) yang buruk untuk penelitian ilmiah.
    \item {Kompresi Acak}: Kompresi (F-04) bersifat acak dan tidak terukur. Satu video mungkin dikompresi oleh WhatsApp, yang lain oleh TikTok, masing-masing dengan artefak yang berbeda. Ini mustahil untuk direplikasi atau dipelajari secara sistematis.
    \item {Keseimbangan Data}: Keseimbangan data (F-07) akan sangat bergantung pada apa yang berhasil ditemukan (\textit{sampling bias}). Kemungkinan besar \textit{dataset} akan sangat tidak seimbang, baik antara jumlah video asli dan palsu, maupun representasi etnisnya (F-02) yang mungkin hanya terkonsentrasi pada beberapa tokoh publik saja.
\end{enumerate}

\subsubsection{Alternatif Solusi 3: Pengembangan \textit{Dataset End-to-End} Konteks Lokal}
Solusi ini merupakan pendekatan paling komprehensif dan satu-satunya yang selaras dengan metodologi \textit{data-centric}. Pendekatan ini meniru metodologi \textit{benchmark} global seperti FaceForensics++ namun menerapkannya pada konteks Indonesia. Proses dimulai dengan Fase Pengumpulan (F-01), di mana video asli lokal yang merepresentasikan wajah dan demografi Indonesia (F-02) dikumpulkan secara strategis. Selanjutnya, pada Fase Generasi (F-03), video-video ini digunakan untuk membuat \textit{deepfake} baru secara internal. Terakhir, baik video asli maupun video palsu yang dihasilkan akan melalui proses simulasi kompresi yang sistematis dan terukur (F-04).

Kelebihan utama dari pendekatan ini adalah {kontrol penuh} atas seluruh \textit{pipeline}.
\begin{enumerate}
    \item {Kontrol Pengumpulan}: Memastikan F-01 (Konteks Lokal) dan F-02 (Representasi Etnis) terpenuhi sesuai strategi pengambilan sampel pada Tabel \ref{tab:suku-bangsa}.
    \item {Kontrol Generasi}: Memastikan F-03 (Teknik Relevan) terpenuhi dan F-06 (Metadata Lengkap) tercatat dengan akurat (misal: video A dibuat dengan \textit{face swap}, video B dengan \textit{lip-sync}).
    \item {Kontrol Kompresi}: Memastikan F-04 (Simulasi Kompresi) diterapkan secara konsisten, memungkinkan pembuatan beberapa versi dari video yang sama (misal: Kualitas Tinggi, Kualitas Rendah 360p, Kualitas Rendah 144p) untuk analisis yang terisolasi.
    \item {Kontrol Pelabelan}: Karena kita memiliki \textit{ground truth} (video asli) untuk setiap \textit{deepfake} yang kita buat, F-05 (Pelabelan Jelas) dan NF-03 (Validitas) terjamin 100\%. Ini juga memungkinkan F-07 (Keseimbangan Data) untuk diatur secara presisi.
\end{enumerate}
Meskipun solusi ini paling padat karya dan membutuhkan sumber daya komputasi yang signifikan (Risiko 3), ini adalah satu-satunya alternatif yang dapat memenuhi seluruh kebutuhan fungsional yang telah diidentifikasi dan menghasilkan \textit{dataset} yang valid secara ilmiah.

\subsection{Analisis Penentuan Solusi}
Berikut ini adalah Tabel \ref{tab:analisis-solusi} yang menampilkan beberapa alternatif solusi yang diajukan untuk memenuhi kebutuhan akan \textit{dataset} \textit{deepfake} berkualitas rendah di Indonesia.

\begin{table}[H] 
    \centering
    \caption{Analisis Penentuan Solusi Dataset}
    \label{tab:analisis-solusi}
    % Menambah sedikit jarak antar baris agar tabel tidak terlalu padat
    \renewcommand{\arraystretch}{1.2} 
    \resizebox{\textwidth}{!}{%
        \begin{tabular}{|l|c|c|c|} 
            \hline
            \textbf{Kebutuhan Fungsional} & \textbf{Alternatif Solusi 1} & \textbf{Alternatif Solusi 2} & \textbf{Alternatif Solusi 3} \\
            \hline
            F-01 (Konteks Lokal) & Tidak Terpenuhi & Terpenuhi & Terpenuhi \\
            \hline
            F-02 (Representasi Etnis) & Tidak Terpenuhi & Tidak Terkontrol & Terpenuhi \\
            \hline
            F-03 (Teknik Relevan) & Terpenuhi & Tidak Diketahui & Terpenuhi \\
            \hline
            F-04 (Simulasi Kompresi) & Terpenuhi & Kompresi Acak & Terpenuhi \\
            \hline
            F-05 (Pelabelan Jelas) & Terpenuhi & Tidak Terkontrol & Terpenuhi \\
            \hline
            F-06 (Metadata Lengkap) & Tidak Terpenuhi & Tidak Diketahui & Terpenuhi \\
            \hline
            F-07 (Keseimbangan Data) & Tergantung Dataset & Tidak Seimbang & Terpenuhi \\
            \hline
        \end{tabular}%
    }
\end{table}

Berdasarkan Tabel \ref{tab:analisis-solusi}, dapat disimpulkan bahwa alternatif solusi 3 (Mengumpulkan video asli lokal, membuat \textit{deepfake} sendiri, dan menerapkan kompresi secara sistematis) adalah satu-satunya yang dapat memenuhi semua kebutuhan fungsional yang diidentifikasi. Solusi ini memberikan kontrol penuh atas konteks, jenis manipulasi, tingkat kompresi, dan pelabelan, yang sangat penting untuk tujuan penelitian.