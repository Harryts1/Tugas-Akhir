% ============================================================================================
% BAB III ANALISIS MASALAH
% Pembagian subbab tidak rigid dan dapat bervariasi. Bab ini minimal berisi analisis kebutuhan
% fungsional dan nonfungsional, analisis berbagai alternatif solusi yang dapat ditawarkan, dan
% metode pemilihan solusi yang diusulkan.
% ============================================================================================
\chapter{ANALISIS MASALAH}
\label{chap:analisis-masalah}
\section{Analisis Kondisi Saat Ini}
Sistem deteksi \textit{deepfake} yang ada saat ini dapat dimodelkan sebagai sebuah alur proses yang bergantung penuh pada ketersediaan data pelatihan. Gambar \ref{gambar:III-1} menggambarkan model konseptual dari alur kerja deteksi \textit{deepfake} yang umum digunakan saat ini di Indonesia.
\begin{figure}[h!]
    \centering
    \includegraphics[width=0.8\textwidth]{image/GambarIII.png}
    \caption{Model Konseptual Alur Deteksi \textit{Deepfake} Saat Ini}
    \label{gambar:III-1}
\end{figure}

Alur kerja dari model konseptual dimulai dari \textit{dataset} publik yang telah banyak ditinjau dalam studi literatur, seperti FaceForensics , Celeb-DFv2 , dan DFDC \autocite{sohan2023}. Sebagian besar dataset ini memiliki karakteristik utama berupa wajah non-Indonesia dan video dengan resolusi yang masih tergolong tinggi. Dataset ini yang kemudian digunakan untuk proses pelatihan , dilanjutkan dengan arsitektur \textit{deep learning} belajar mengenali pola dan artefak manipulasi. Hasil dari proses ini adalah sebuah model deteksi yang menunjukkan performa sangat tinggi di lingkungan laboratorium.

Namun, seperti yang telah diidentifikasi dalam studi literatur , masalah fundamental muncul apabila model tersebut digunakan untuk mendeteksi \textit{deepfake} dalam konteks dunia nyata di Indonesia. Video-video di Indonesia berisi orang-orang lokal (Indonesia) dan, yang lebih penting, mengalami kompresi agresif akibat penyebaran di sosial media serta \textit{bandwidth} internet yang tidak merata.

Kondisi ini sejalan dengan temuan \autocite{montibeller2025}, yang menyatakan bahwa kompresi video secara signifikan merusak fitur forensik tingkat rendah, sehingga menyulitkan detektor. Akibat dari kesenjangan data (distribusi \textit{out-of-distribution}) antara data latih (resolusi tinggi, non-Indonesia) dengan data di lapangan (resolusi rendah, Indonesia) , performa model yang sebelumnya tinggi dapat menurun drastis. Kondisi inilah yang menjadi justifikasi utama perlunya pengembangan \textit{dataset} baru yang dapat menjembatani kesenjangan tersebut.

\section{Analisis Kebutuhan}
Berikut ini adalah penjabaran terkait kebutuhan-kebutuhan untuk membuat \textit{dataset} yang baru.

\subsection{Identifikasi Masalah Pengguna}
Pengguna sistem ini adalah Steven (pembuat model) dan peneliti lain di bidang deteksi \textit{deepfake}. Masalah yang dihadapi oleh para pengguna adalah:
\begin{enumerate}
    \item Steven (pembuat model)
        \begin{enumerate}[label=\alph*.]
            \item Steven membutuhkan data pelatihan yang memadai untuk membuat model yang \textit{robust} terhadap video berkualitas rendah di Indonesia.
            \item Modelnya mungkin akan \textit{overfitting} pada video berkualitas tinggi dan gagal di skenario dunia nyata Indonesia.
        \end{enumerate}
    \item Peneliti lainnya
        \begin{enumerate}[label=\alph*.]
            \item Peneliti tidak memiliki \textit{benchmark dataset} untuk menguji validitas model deteksi \textit{deepfake} mereka dalam konteks Indonesia.
        \end{enumerate}
\end{enumerate}

Untuk mencari solusi atas masalah-masalah tersebut, perlu disusun kebutuhan fungsional dan nonfungsional sistem yang diperlukan. Subbab berikut menjabarkan kebutuhan-kebutuhan tersebut.

\subsection{Kebutuhan Fungsional}
Berikut ini adalah Tabel \ref{tab:kebutuhan-fungsional} yang berisi kebutuhan-kebutuhan fungsional yang dibutuhkan untuk menjawab masalah-masalah tersebut.

\begin{table}[H] % Menggunakan [H] dari paket float untuk memaksa posisi
    \centering
    \caption{Kebutuhan Fungsional Dataset}
    \label{tab:kebutuhan-fungsional}
    \begin{tabular}{|l|l|p{8cm}|} % l=rata kiri, p{lebar}=paragraf rata kiri
        \hline
        \textbf{Kode} & \textbf{Kebutuhan Fungsional} & \textbf{Deskripsi} \\
        \hline
        F-01 & Konteks Lokal & \textit{Dataset} harus berisi video asli yang menampilkan wajah orang Indonesia yang beragam. \\
        \hline
        F-02 & Representasi Etnis & \textit{Dataset} harus merepresentasikan keragaman suku bangsa utama di Indonesia. \\
        \hline
        F-03 & Teknik Relevan & \textit{Dataset} harus berisi video palsu yang dibuat menggunakan teknik \textit{deepfake} yang umum, seperti visual \textit{deepfake}, audio \textit{deepfake}, maupun \textit{multimodal deepfake}. \\
        \hline
        F-04 & Simulasi Kompresi & \textit{Dataset} harus menyertakan versi video (asli dan palsu) yang telah disimulasikan efek kompresinya untuk meniru kualitas rendah di media sosial. \\
        \hline
        F-05 & Pelabelan Jelas & Setiap video dalam \textit{dataset} harus memiliki label yang jelas. \\
        \hline
        F-06 & Metadata Lengkap & \textit{Dataset} harus disertai \textit{metadata} yang detail untuk setiap video palsu. \\
        \hline
        F-07 & Keseimbangan Data & \textit{Dataset} harus memiliki keseimbangan yang wajar antara jumlah video asli dan palsu, serta antara video berkualitas tinggi dan rendah. \\
        \hline
    \end{tabular}
\end{table}

\subsection{Kebutuhan Nonfungsional}
Berikut ini adalah Tabel \ref{tab:kebutuhan-non-fungsional} yang berisi atribut-atribut mengenai bagaimana (\textit{HOW}) sistem bekerja.

\begin{table}[H] % Menggunakan [H] dari paket float untuk memaksa posisi
    \centering
    \caption{Kebutuhan Non Fungsional Dataset}
    \label{tab:kebutuhan-non-fungsional}
    \begin{tabular}{|l|l|p{8cm}|} % l=rata kiri, p{lebar}=paragraf rata kiri
        \hline
        \textbf{Kode} & \textbf{Kebutuhan Non Fungsional} & \textbf{Penjelasan} \\
        \hline
        NF-01 & Relevansi & \textit{Dataset} harus sangat relevan dengan masalah disinformasi \textit{deepfake} di Indonesia. \\
        \hline
        NF-02 & Keberagaman & \textit{Dataset} harus beragam, tidak hanya dari segi etnis, tetapi juga pencahayaan, latar belakang, dan usia subjek. \\
        \hline
        NF-03 & Validitas & Label harus 100\% akurat. Proses validasi harus memastikan tidak ada kesalahan pelabelan. \\
        \hline
        NF-04 & Realisme & Video berkualitas rendah yang dihasilkan harus secara akurat mencerminkan artefak kompresi yang ditemukan di platform nyata. \\
        \hline
    \end{tabular}
\end{table}

\section{Analisis Pemilihan Solusi}
Berikut ini adalah beberapa alternatif solusi yang diusulkan beserta metode dalam penentuan solusi.

\subsection{Alternatif Solusi}
Berikut ini adalah Tabel \ref{tab:alternatif-solusi} yang menampilkan beberapa alternatif solusi yang diajukan untuk memenuhi kebutuhan akan \textit{dataset} \textit{deepfake} berkualitas rendah di Indonesia.

\begin{table}[H] % Menggunakan [H] dari paket float untuk memaksa posisi
    \centering
    \caption{Alternatif Solusi Dataset}
    \label{tab:alternatif-solusi}
    \begin{tabular}{|c|p{10cm}|} % c=rata tengah, p{lebar}=paragraf rata kiri
        \hline
        \textbf{Alternatif Solusi} & \textbf{Solusi} \\
        \hline
        1 & Mengambil \textit{dataset} yang ada dan hanya menerapkan simulasi kompresi pada video-video tersebut. \\
        \hline
        2 & Hanya mengumpulkan video \textit{deepfake} yang sudah beredar di media sosial Indonesia. \\
        \hline
        3 & Mengumpulkan video asli lokal, membuat \textit{deepfake} sendiri, dan menerapkan kompresi secara sistematis. \\
        \hline
    \end{tabular}
\end{table}

\subsection{Analisis Penentuan Solusi}
Berikut ini adalah Tabel \ref{tab:analisis-solusi} yang menampilkan beberapa alternatif solusi yang diajukan untuk memenuhi kebutuhan akan \textit{dataset} \textit{deepfake} berkualitas rendah di Indonesia.

\begin{table}[H] % Menggunakan [H] dari paket float untuk memaksa posisi
    \centering
    \caption{Analisis Penentuan Solusi Dataset}
    \label{tab:analisis-solusi}
    \begin{tabular}{|l|c|c|c|} % l=rata kiri, c=rata tengah
        \hline
        \textbf{Kebutuhan Fungsional} & \textbf{Alternatif Solusi 1} & \textbf{Alternatif Solusi 2} & \textbf{Alternatif Solusi 3} \\
        \hline
        F-01 (Konteks Lokal) & Tidak Terpenuhi & Terpenuhi & Terpenuhi \\
        \hline
        F-02 (Representasi Etnis) & Tidak Terpenuhi & Tidak Terkontrol & Terpenuhi \\
        \hline
        F-03 (Teknik Relevan) & Terpenuhi & Tidak Diketahui & Terpenuhi \\
        \hline
        F-04 (Simulasi Kompresi) & Terpenuhi & Kompresi Acak & Terpenuhi \\
        \hline
        F-05 (Pelabelan Jelas) & Terpenuhi & Tidak Terkontrol & Terpenuhi \\
        \hline
        F-06 (Metadata Lengkap) & Tidak Terpenuhi & Tidak Diketahui & Terpenuhi \\
        \hline
        F-07 (Keseimbangan Data) & Tergantung Dataset & Tidak Seimbang & Terpenuhi \\
        \hline
    \end{tabular}
\end{table}

Berdasarkan Tabel \ref{tab:analisis-solusi}, dapat disimpulkan bahwa alternatif solusi 3 (Mengumpulkan video asli lokal, membuat \textit{deepfake} sendiri, dan menerapkan kompresi secara sistematis) adalah satu-satunya yang dapat memenuhi semua kebutuhan fungsional yang diidentifikasi. Solusi ini memberikan kontrol penuh atas konteks, jenis manipulasi, tingkat kompresi, dan pelabelan, yang sangat penting untuk tujuan penelitian.